% % @author Shuning Zhang
% % @date 2020-05-01
% \documentclass[a4paper, 11pt]{ctexart}
% \usepackage{amsfonts, amsmath, amssymb, amsthm}
% \usepackage{color}
% \usepackage{enumerate}
% \usepackage[bottom=2cm, left=2.5cm, right=2.5cm, top=2cm]{geometry}
% \usepackage{multicol}
% \begin{document}
\begin{enumerate}
    \item % 1
        \begin{enumerate}[(1)]
            \item % 1.1
                {\color{red}remained}
            \item % 1.2
                显然是, 因为所有 $n\times n$ 的上三角实矩阵组成的集合是所有 $n\times n$ 实矩阵组成的集合的子集, 而 $n\times n$ 实矩阵组成的集合是一个线性空间, 其子集当然是一个线性空间.
            \item % 1.3
                可导必连续, 连续不一定可导, 因此 $[0, 1]$ 上全体可导函数构成的集合是 $C[0, 1]$ 的子集, 又 $C[0, 1]$ 是线性空间, 因此 $[0, 1]$ 上全体可导函数构成的集合是一个线性空间.
            \item % 1.4
                是.
        \end{enumerate}
    \item % 2
        \begin{enumerate}[(1)]
            \item % 2.1
                $-(-\alpha) = ((-1)\cdot(-1))\alpha = \alpha$;
            \item % 2.2
                $-(k\alpha) = ((-1)k)\alpha = (k(-1))\alpha = k((-1)\alpha) = k(-\alpha)$;
            \item % 2.3
                $k(\alpha-\beta) = k(\alpha+(-\beta)) = k\alpha + k(-\beta) = k\alpha + (k(-1))\beta = k\alpha + ((-1)k)\beta = k\alpha - k\beta$.
        \end{enumerate}
\end{enumerate}
% \end{document}
