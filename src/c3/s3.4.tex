% % @author Shuning Zhang
% % @date 2020-05-03
% \documentclass[a4paper, 11pt]{ctexart}
% \usepackage{amsfonts, amsmath, amssymb, amsthm}
% \usepackage{color}
% \usepackage{enumerate}
% \usepackage[bottom=2cm, left=2.5cm, right=2.5cm, top=2cm]{geometry}
% \usepackage{multicol}
% \begin{document}
\begin{enumerate}
    \item % 1
        \begin{multicols}{2}
            \begin{enumerate}[(1)]
                \item % 2.1
                    线性相关;
                \item % 2.2
                    线性无关.
            \end{enumerate}
        \end{multicols}
    \item % 2
        线性相关, 即这三个向量组成的行列式等于零, 那么
        \begin{align*}
            \left|
                \begin{array}{ccc}
                    a & -\frac12 & -\frac12 \\
                    -\frac12 & a & -\frac12 \\
                    -\frac12 & -\frac12 & a
                \end{array}
            \right|
            &=
            \left|
                \begin{array}{ccc}
                    a+\frac12 & -\frac12 & -\frac12 \\
                    0 & a & -\frac12 \\
                    -a-\frac12 & -\frac12 & a
                \end{array}
            \right|
            =
            \left|
                \begin{array}{ccc}
                    a+\frac12 & -\frac12 & -\frac12 \\
                    0 & a & -\frac12 \\
                    0 & -1 & a-\frac12
                \end{array}
            \right| \\
            &=
            \left(a+\frac12\right)
            \left|
                \begin{array}{cc}
                    a & -\frac12 \\
                    -1 & a-\frac12
                \end{array}
            \right|
            =
            \left(a+\frac12\right)\left(a^2-\frac12a-\frac12\right)
            = 0.
        \end{align*}
        解得 $a = -\dfrac12$ 或 $a = 1$.
    \item % 3
        否.
    \item % 4
        否.
    \item % 5
        取 $k_1, k_2, \cdots, k_m \in \mathrm{K}$, 令
        \[
            k_1(c\alpha_1) + k_2(c\alpha_2) + \cdots + k_m(c\alpha_m) = 0,    
        \]
        即
        \[
            c(k_1\alpha_1 + k_2\alpha_2 + \cdots + k_m\alpha_m) = 0.   
        \]
        因 $c \not= 0$, 故
        \[
            k_1\alpha_1 + k_2\alpha_2 + \cdots + k_m\alpha_m = 0,   
        \]
        又因 $\alpha_1, \alpha_2, \cdots, \alpha_m$ 线性无关, 故 $k_1 = k_2 = \cdots = k_m = 0$.
    \item % 6
        否.
    \item % 7
        \begin{proof}
            若 $\alpha_1, \alpha_2, \cdots, \beta$ 线性无关, 命题得证.

            若 $\alpha_1, \alpha_2, \cdots, \alpha_m, \beta$ 线性相关. 取 $k_1, \cdots, k_{m+1} \in \mathrm{K}$, 令
            \[
                k_1\alpha_1 + k_2\alpha_2 + \cdots + k_{m}\alpha_m + k_{m+1}\beta = 0,    
            \]
            若 $k_{m+1} = 0$, 即 $k_1\alpha_1 + k_2\alpha_2 + \cdots + k_{m}\alpha_m = 0$, 因 $\alpha_1, \cdots, \alpha_m$ 线性无关, 所以 $k_1 = \cdots = k_m = 0$.
            表明 $\alpha_1, \cdots, \alpha_m, \beta$ 线性无关, 与前提矛盾. 因此必有 $k_{m+1} \not= 0$, 则有
            \[
                \beta = -\frac{k_1}{k_{m+1}}\alpha_1 - \cdots -\frac{k_{m}}{k_{m+1}}\alpha_m,   
            \]
            即 $\beta$ 可由 $\alpha_1, \alpha_2, \cdots, \alpha_m$ 线性表示.
        \end{proof}
    \item % 8
        \begin{proof}
            取 $k_1, k_2, \cdots, k_m \in \mathrm{K}$, 令
            \[
                k_1(A\alpha_1) + k_2(A\alpha_2) + \cdots + k_m(A\alpha_m) = 0,  
            \]
            根据矩阵乘法的运算法则, 则有
            \[
                A(k_1\alpha_1) + A(k_2\alpha_2) + \cdots + A(k_m\alpha_m) = 0,    
            \]
            在上式两边同乘以 $A^{-1}$, 则有
            \[
                k_1\alpha_1 + k_2\alpha_2 + \cdots + k_m\alpha_m = 0.    
            \]
            因 $\alpha_1, \alpha_2, \cdots, \alpha_m$ 线性无关, 故 $k_1 = k_2 = \cdots = k_m = 0$.
        \end{proof}
    \item % 9
        \begin{proof}
            将 $\alpha_1x_1 + \alpha_2x_2 + \cdots + \alpha_rx_r = 0$ 看作一个有 $n$ 个方程, $r$ 个未知量的齐次线性方程组.
            那么
            \begin{equation}
                \widetilde{\alpha}_1x_1 + \widetilde{\alpha}_2x_2 + \cdots + \widetilde{\alpha}_rx_r = 0
            \end{equation}
            便是由原方程组减少 $n-j$ 个方程组成的方程组, 即 $j\ (j < n)$ 个方程, $r$ 个未知量的齐次线性方程组, 已知 $\widetilde{\alpha}_1, \widetilde{\alpha}_2, \cdots, \widetilde{\alpha}_r$ 线性无关, 即这 $j$ 个方程组成的方程组只有零解.
            因此再添加 $n-j$ 个方程不会影响解, 因此 $\alpha_1x_1 + \alpha_2x_2 + \cdots + \alpha_rx_r = 0$ 也只有零解, 即 $\alpha_1, \cdots, \alpha_r$ 也线性无关. 
        \end{proof}
    \item % 10
        \begin{proof}
            用反证法. 假设 $\alpha_1, \alpha_2, \cdots, \alpha_m$ 线性相关, 由定理 3.4.2 可知其中至少有一个向量可以由其余向量线性表示, 不妨假定这个向量为 $\alpha_1$.
            即
            \[
                \alpha_1 = l_2\alpha_2 + \cdots + l_m\alpha_m,    
            \]
            又由已知条件 $\beta$ 可由 $\alpha_1, \cdots, \alpha_m$ 线性表示, 即
            \[
                \beta = k_1\alpha_1 + \cdots + k_m\alpha_m,    
            \]
            将 $\alpha_1$ 的线性表示带入上式, 有
            \begin{align*}
                \beta &= k_1(l_2\alpha_2 + \cdots + l_m\alpha_m) + k_2\alpha_2 + \cdots + k_m\alpha_m \\
                &=  (k_1l_2 + k_2)\alpha_2 + \cdots + (k_1l_m + k_m)\alpha_m  
            \end{align*}
            即 $\beta$ 可由 $\alpha_2, \cdots, \alpha_m$ 这 $m - 1$ 个向量线性表示, 与题意矛盾. 因此 $\alpha_1, \cdots, \alpha_m$ 线性无关.
        \end{proof}
    \item % 11
        \begin{proof}
            记 $B = (\beta_1, \beta_2, \cdots, \beta_n)$, 其中 $\beta_i$ 是 $m\times1$ 矩阵, 那么
            \[
                AB = (A\beta_1, \cdots, A\beta_n) = I_n = (e_1, \cdots, e_n),    
            \]
            即 $A\beta_i = e_i$. 考虑
            \[
                k_1\beta_1 + k_2\beta_2 + \cdots + k_n\beta_n = 0,
            \]
            在上式两边同乘以 $A$, 有
            \begin{align*}
                0 = A(k_1\beta_1 + \cdots + k_n\beta_n) &= k_1A\beta_1 + \cdots + k_nA\beta_n \\
                &= k_1e_1 + \cdots + k_ne_n = (k_1, \cdots, k_n)',  
            \end{align*}
            即 $k_1 = k_2 = \cdots = k_n = 0$, 因此 $\beta_1, \cdots, \beta_n$ 线性无关.
        \end{proof}
\end{enumerate}
% \end{document}
