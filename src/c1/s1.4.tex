% % @author Shuning Zhang
% % @date 2020-03-17
% \documentclass[a4paper, 11pt]{ctexart}
% \usepackage{amsfonts, amsmath, amssymb, amsthm}
% \usepackage{color}
% \usepackage{enumerate}
% \usepackage[bottom=2cm, left=2.5cm, right=2.5cm, top=2cm]{geometry}
% \usepackage{multicol}
% \begin{document}
\begin{enumerate}
    \item % 1
        略.
    \item % 2
        略.
    \item % 3
        \begin{proof}
            \begin{align*}
                |A| &=
                \left|
                    \begin{array}{cccc}
                        0 & a_{12} & \cdots & a_{1n} \\
                        -a_{12} & 0 & \cdots & a_{2n} \\
                        \vdots & \vdots & \ddots & \vdots \\
                        -a_{1n} & -a_{2n} & \cdots & 0
                    \end{array}
                \right| \\
                &= (-1)
                \left|
                    \begin{array}{cccc}
                        0 & -a_{12} & \cdots & -a_{1n} \\
                        -a_{12} & 0 & \cdots & a_{2n} \\
                        \vdots & \vdots & \ddots & \vdots \\
                        -a_{1n} & -a_{2n} & \cdots & 0
                    \end{array}
                \right| \\
                &= (-1)^n
                \left|
                    \begin{array}{cccc}
                        0 & -a_{12} & \cdots & -a_{1n} \\
                        a_{12} & 0 & \cdots & -a_{2n} \\
                        \vdots & \vdots & \ddots & \vdots \\
                        a_{1n} & a_{2n} & \cdots & 0
                    \end{array}
                \right| \\
                &= (-1)^n|A'|.
            \end{align*}
            因为 $n$ 为奇数, 故 $|A| = -|A'|$, 因此 $|A| = 0$.
        \end{proof}
    \item % 4
        \begin{proof}
            \begin{align*}
                \left|
                    \begin{array}{ccccc}
                        0 & 0 & \cdots & 0 & b_1 \\
                        0 & 0 & \cdots & b_2 & 0 \\
                        \vdots & \vdots & \ddots & \vdots & \vdots \\
                        0 & b_{n-1} & \cdots & 0 & 0 \\
                        b_n & 0 & \cdots & 0 & 0
                    \end{array}
                \right|
                &= (-1)^{n-1}
                \left|
                    \begin{array}{ccccc}
                        b_1 & 0 & \cdots & 0 & 0 \\
                        0 & 0 & \cdots & 0 & b_2 \\
                        \vdots & \vdots & \ddots & \vdots & \vdots \\
                        0 & 0 & \cdots & 0 & 0 \\
                        0 & b_n & \cdots & 0 & 0
                    \end{array}
                \right| \\
                &= (-1)^{n-1}(-1)^{n-2} 
                \left|
                    \begin{array}{ccccc}
                        b_1 & 0 & \cdots & 0 & 0 \\
                        0 & b_2 & \cdots & 0 & 0 \\
                        \vdots & \vdots & \ddots & \vdots & \vdots \\
                        0 & 0 & \cdots & 0 & 0 \\
                        0 & 0 & \cdots & 0 & 0
                    \end{array}
                \right| \\
                &= (-1)^{\frac{(n-1)n}{2}}
                \left|
                    \begin{array}{ccccc}
                        b_1 & 0 & \cdots & 0 & 0 \\
                        0 & b_2 & \cdots & 0 & 0 \\
                        \vdots & \vdots & \ddots & \vdots & \vdots \\
                        0 & 0 & \cdots & b_{n-1} & 0 \\
                        0 & 0 & \cdots & 0 & b_n
                    \end{array}
                \right| \\
                &= (-1)^{\frac{(n-1)n}{2}}b_1b_2 \cdots b_n. \qedhere   
            \end{align*}
        \end{proof}
    \item % 5
        将行列式按第 $2$ 列进行展开, 则有
        \[
            f(x) = x
            \left|
                \begin{array}{cccc}
                    1 & 3 & 4 \\
                    -1 & -2 & -3 \\
                    -1 & -2 & -2
                \end{array}
            \right|
            + 2 \cdot A_{22} + 0 \cdot A_{32} + 7 \cdot A_{42}.    
        \]
        因此 $x$ 的系数为
        \[
            \left|
                \begin{array}{cccc}
                    1 & 3 & 4 \\
                    -1 & -2 & -3 \\
                    -1 & -2 & -2
                \end{array}
            \right|
            =
            \left|
                \begin{array}{cccc}
                    1 & 3 & 4 \\
                    0 & 1 & 1 \\
                    0 & 1 & 2
                \end{array}
            \right|
            =
            \left|
                \begin{array}{cccc}
                    1 & 3 & 4 \\
                    0 & 1 & 1 \\
                    0 & 0 & 1
                \end{array}
            \right|
            = 1. 
        \]
    \item % 6
        略.
\end{enumerate}
% \end{document}
