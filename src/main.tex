% @author Shuning Zhang
% @date 2020-03-15

\documentclass[a4paper, 11pt]{ctexbook}

\usepackage{amsfonts, amsmath, amssymb, amsthm}
\usepackage{color}
\usepackage{enumerate}
\usepackage[bottom=2cm, left=2.5cm, right=2.5cm, top=2cm]{geometry}
\usepackage[bookmarksnumbered=true, hidelinks]{hyperref}
\usepackage{multicol}

\begin{document}
    \chapter{行列式}
        \section{二阶行列式}
            % % @author Shuning Zhang
% % @date 2020-03-16
% \documentclass[a4paper, 11pt]{ctexart}
% \usepackage{amsfonts, amsmath, amssymb, amsthm}
% \usepackage{color}
% \usepackage{enumerate}
% \usepackage[bottom=2cm, left=2.5cm, right=2.5cm, top=2cm]{geometry}
% \usepackage{multicol}
% \begin{document}
\begin{enumerate}
    \item % 1
        \begin{enumerate}[(1)]
            \item % 1.1
                $-2$;
            \item % 1.2
                $1$.
        \end{enumerate}
    \item % 2
        \begin{enumerate}[(1)]
            \item % 2.1
                $
                    \left|\begin{array}{cccc}
                        2 & 1 \\
                        -4 & 4
                    \end{array}\right|
                    =
                    4\left|
                        \begin{array}{cccc}
                            2 & 1 \\
                            -1 & 1
                        \end{array}
                    \right|    
                $;
            \item % 2.2
                $
                    \left|
                        \begin{array}{cccc}
                            2 & 3 \\
                            -1 & 3
                        \end{array}
                    \right|
                    =
                    3\left|
                        \begin{array}{cccc}
                            2 & 1 \\
                            -1 & 1
                        \end{array}
                    \right|
                $.
        \end{enumerate}
    \item % 3
        $
            \left|
                \begin{array}{cccc}
                    3 & 2 \\
                    4 & 3
                \end{array}
            \right|
            =
            \left|
                \begin{array}{cccc}
                    3 & 2 \\
                    3 & 1
                \end{array}
            \right|
            +
            \left|
                \begin{array}{cccc}
                    3 & 2 \\
                    1 & 2
                \end{array}
            \right|
        $.
    \item % 4
        $
            \left|
                \begin{array}{cccc}
                    5 & -2 \\
                    2 & 1
                \end{array}
            \right|
            =
            -\left|
                \begin{array}{cccc}
                    -2 & 1 \\
                    5 & 2
                \end{array}
            \right|
        $.
    \item % 5
        $
            \left|
                \begin{array}{cccc}
                    2 & 3 \\
                    -3 & 1
                \end{array}
            \right|
            =
            \left|
                \begin{array}{cccc}
                    2 & -3 \\
                    3 & 1
                \end{array}
            \right|
        $.
    \item % 6
        $
            \left|
                \begin{array}{cccc}
                    a_{11} + b_{11} & a_{12} + b_{12} \\
                    a_{21} + b_{21} & a_{22} + b_{22}
                \end{array}
            \right|
            =
            \left|
                \begin{array}{cccc}
                    a_{11} & a_{12} \\
                    a_{21} & a_{22}
                \end{array}
            \right|
            +
            \left|
                \begin{array}{cccc}
                    a_{11} & b_{12} \\
                    a_{21} & b_{22}
                \end{array}
            \right|
            +
            \left|
                \begin{array}{cccc}
                    b_{11} & a_{12} \\
                    b_{21} & a_{22}
                \end{array}
            \right|
            +
            \left|
                \begin{array}{cccc}
                    b_{11} & b_{12} \\
                    b_{21} & b_{22}
                \end{array}
            \right|
        $.
\end{enumerate}
% \end{document}

        \section{三阶行列式}
            % % @author Shuning Zhang
% % @date 2020-03-16
% \documentclass[a4paper, 11pt]{ctexart}
% \usepackage{amsfonts, amsmath, amssymb, amsthm}
% \usepackage{color}
% \usepackage{enumerate}
% \usepackage[bottom=2cm, left=2.5cm, right=2.5cm, top=2cm]{geometry}
% \usepackage{multicol}
% \begin{document}
\begin{enumerate}
    \item % 1
        \begin{enumerate}[(1)]
            \item % 1.1
                $-8$;
            \item % 1.2
                $11$.
        \end{enumerate}
    \item % 2
        \begin{enumerate}[(1)]
            \item % 2.1
                $0$;
            \item % 2.2
                $22$.
        \end{enumerate}
    \item % 3
        \begin{enumerate}[(1)]
            \item % 3.1
                $0$;
            \item % 3.2
                $(x^2+1)e^x - x$.
        \end{enumerate}
    \item % 4
        \begin{enumerate}[(1)]
            \item % 4.1
                $2$;
            \item % 4.2
                $13$.
        \end{enumerate}
    \item % 5
        \begin{enumerate}[(1)]
            \item % 5.1
                $x_1 = -\dfrac{13}{5}$, $x_2 = -\dfrac{4}{5}$, $x_3 = -\dfrac{7}{5}$;
            \item % 5.2
                $x = 2$, $y = 0$, $z = -2$.
        \end{enumerate}
\end{enumerate}
% \end{document}

        \section{\texorpdfstring{$n$}{n} 阶行列式}
            % % @author Shuning Zhang
% % @date 2020-03-16
% \documentclass[a4paper, 11pt]{ctexart}
% \usepackage{amsfonts, amsmath, amssymb, amsthm}
% \usepackage{color}
% \usepackage{enumerate}
% \usepackage[bottom=2cm, left=2.5cm, right=2.5cm, top=2cm]{geometry}
% \usepackage{multicol}
% \begin{document}
\begin{enumerate}
    \item % 1
        略.
    \item % 2
        \begin{enumerate}[(1)]
            \item % 2.1
                $abcd$;
            \item % 2.2
                $-6$.
        \end{enumerate}
    \item % 3
        \begin{enumerate}[(1)]
            \item % 3.1
                $60$;
            \item % 3.2
                $2$.
        \end{enumerate}
    \item % 4
        \begin{enumerate}[(1)]
            \item % 4.1
                $0$;
            \item % 4.2
                $0$.
        \end{enumerate}
\end{enumerate}
% \end{document}

        \section{行列式的展开与转置}
            % % @author Shuning Zhang
% % @date 2020-03-17
% \documentclass[a4paper, 11pt]{ctexart}
% \usepackage{amsfonts, amsmath, amssymb, amsthm}
% \usepackage{color}
% \usepackage{enumerate}
% \usepackage[bottom=2cm, left=2.5cm, right=2.5cm, top=2cm]{geometry}
% \usepackage{multicol}
% \begin{document}
\begin{enumerate}
    \item % 1
        略.
    \item % 2
        略.
    \item % 3
        \begin{proof}
            \begin{align*}
                |A| &=
                \left|
                    \begin{array}{cccc}
                        0 & a_{12} & \cdots & a_{1n} \\
                        -a_{12} & 0 & \cdots & a_{2n} \\
                        \vdots & \vdots & \ddots & \vdots \\
                        -a_{1n} & -a_{2n} & \cdots & 0
                    \end{array}
                \right| \\
                &= (-1)
                \left|
                    \begin{array}{cccc}
                        0 & -a_{12} & \cdots & -a_{1n} \\
                        -a_{12} & 0 & \cdots & a_{2n} \\
                        \vdots & \vdots & \ddots & \vdots \\
                        -a_{1n} & -a_{2n} & \cdots & 0
                    \end{array}
                \right| \\
                &= (-1)^n
                \left|
                    \begin{array}{cccc}
                        0 & -a_{12} & \cdots & -a_{1n} \\
                        a_{12} & 0 & \cdots & -a_{2n} \\
                        \vdots & \vdots & \ddots & \vdots \\
                        a_{1n} & a_{2n} & \cdots & 0
                    \end{array}
                \right| \\
                &= (-1)^n|A'|.
            \end{align*}
            因为 $n$ 为奇数, 故 $|A| = -|A'|$, 因此 $|A| = 0$.
        \end{proof}
    \item % 4
        \begin{proof}
            \begin{align*}
                \left|
                    \begin{array}{ccccc}
                        0 & 0 & \cdots & 0 & b_1 \\
                        0 & 0 & \cdots & b_2 & 0 \\
                        \vdots & \vdots & \ddots & \vdots & \vdots \\
                        0 & b_{n-1} & \cdots & 0 & 0 \\
                        b_n & 0 & \cdots & 0 & 0
                    \end{array}
                \right|
                &= (-1)^{n-1}
                \left|
                    \begin{array}{ccccc}
                        b_1 & 0 & \cdots & 0 & 0 \\
                        0 & 0 & \cdots & 0 & b_2 \\
                        \vdots & \vdots & \ddots & \vdots & \vdots \\
                        0 & 0 & \cdots & 0 & 0 \\
                        0 & b_n & \cdots & 0 & 0
                    \end{array}
                \right| \\
                &= (-1)^{n-1}(-1)^{n-2} 
                \left|
                    \begin{array}{ccccc}
                        b_1 & 0 & \cdots & 0 & 0 \\
                        0 & b_2 & \cdots & 0 & 0 \\
                        \vdots & \vdots & \ddots & \vdots & \vdots \\
                        0 & 0 & \cdots & 0 & 0 \\
                        0 & 0 & \cdots & 0 & 0
                    \end{array}
                \right| \\
                &= (-1)^{\frac{(n-1)n}{2}}
                \left|
                    \begin{array}{ccccc}
                        b_1 & 0 & \cdots & 0 & 0 \\
                        0 & b_2 & \cdots & 0 & 0 \\
                        \vdots & \vdots & \ddots & \vdots & \vdots \\
                        0 & 0 & \cdots & b_{n-1} & 0 \\
                        0 & 0 & \cdots & 0 & b_n
                    \end{array}
                \right| \\
                &= (-1)^{\frac{(n-1)n}{2}}b_1b_2 \cdots b_n. \qedhere   
            \end{align*}
        \end{proof}
    \item % 5
        将行列式按第 $2$ 列进行展开, 则有
        \[
            f(x) = x
            \left|
                \begin{array}{cccc}
                    1 & 3 & 4 \\
                    -1 & -2 & -3 \\
                    -1 & -2 & -2
                \end{array}
            \right|
            + 2 \cdot A_{22} + 0 \cdot A_{32} + 7 \cdot A_{42}.    
        \]
        因此 $x$ 的系数为
        \[
            \left|
                \begin{array}{cccc}
                    1 & 3 & 4 \\
                    -1 & -2 & -3 \\
                    -1 & -2 & -2
                \end{array}
            \right|
            =
            \left|
                \begin{array}{cccc}
                    1 & 3 & 4 \\
                    0 & 1 & 1 \\
                    0 & 1 & 2
                \end{array}
            \right|
            =
            \left|
                \begin{array}{cccc}
                    1 & 3 & 4 \\
                    0 & 1 & 1 \\
                    0 & 0 & 1
                \end{array}
            \right|
            = 1. 
        \]
    \item % 6
        略.
\end{enumerate}
% \end{document}

        \section{行列式的计算}
        \section{行列式的等价定义}
        \section{Laplace 定理}
    \chapter{矩阵}
        \section{矩阵的概念}
        \section{矩阵的运算}
        \section{方阵的逆阵}
        \section{矩阵的初等变换与初等矩阵}
        \section{矩阵乘积的行列式与初等变换法求逆阵}
        \section{分块矩阵}
        \section{Cauchy-Binet 公式}
    \chapter{线性空间}
        \section{数域}
        \section{行向量和列向量}
        \section{线性空间}
        \section{向量的线性关系}
        \section{向量组的秩}
        \section{矩阵的秩}
        \section{坐标向量}
        \section{基变换与过渡矩阵}
        \section{子空间}
        \section{线性方程组的解}
    \chapter{线性映射}
        \section{线性映射的概念}
        \section{线性映射的运算}
        \section{线性映射与矩阵}
        \section{线性映射的像与核}
        \section{不变子空间}
    \chapter{多项式}
        \section{一元多项式代数}
        \section{整除}
        \section{最大公因式}
        \section{因式分解}
        \section{多项式函数}
        \section{复系数多项式}
        \section{实系数多项式和有理系数多项式}
        \section{多元多项式}
        \section{对称多项式}
        \section{结式和判别式}
    \chapter{特征值}
        \section{特征值和特征向量}
        \section{对角化}
        \section{极小多项式与 Cayley-Hamilton 定理}
        \section{特征值的估计}
    \chapter{相似标准型}
        \section{多项式矩阵}
        \section{矩阵的法式}
        \section{不变因子}
        \section{有理标准型}
        \section{初等因子}
        \section{Jordan 标准型}
        \section{Jordan 标准型的进一步讨论和应用}
        \section{矩阵函数}
    \chapter{二次型}
        \section{二次型的化简与矩阵的合同}
        \section{二次型的化简}
        \section{惯性定理}
        \section{正定型与正定矩阵}
        \section{Hermite 型}
    \chapter{内积空间}
        \section{内积空间的概念}
        \section{内积的表示和正交基}
        \section{伴随}
        \section{内积空间的同构, 正交变换和酉变换}
        \section{自伴随算子}
        \section{复正规算子}
        \section{实正规矩阵}
        \section{谱}
        \section{奇异值分解}
        \section{最小二乘解}
    \chapter{双线性型}
        \section{对偶空间}
        \section{双线性型}
        \section{纯量积}
        \section{交错型与辛空间}
        \section{对称型与正交几何}
\end{document}
