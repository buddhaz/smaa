% @author Shuning Zhang
% @date 2020-03-15

\documentclass[a4paper, 11pt]{ctexbook}

\usepackage{amsfonts, amsmath, amssymb, amsthm}
\usepackage{color}
\usepackage{enumerate}
\usepackage[bottom=2cm, left=2.5cm, right=2.5cm, top=2cm]{geometry}
\usepackage[bookmarksnumbered=true, hidelinks]{hyperref}
\usepackage{multicol}

\begin{document}
    \chapter{行列式}
        \section{二阶行列式}
            % % @author Shuning Zhang
% % @date 2020-03-16
% \documentclass[a4paper, 11pt]{ctexart}
% \usepackage{amsfonts, amsmath, amssymb, amsthm}
% \usepackage{color}
% \usepackage{enumerate}
% \usepackage[bottom=2cm, left=2.5cm, right=2.5cm, top=2cm]{geometry}
% \usepackage{multicol}
% \begin{document}
\begin{enumerate}
    \item % 1
        \begin{enumerate}[(1)]
            \item % 1.1
                $-2$;
            \item % 1.2
                $1$.
        \end{enumerate}
    \item % 2
        \begin{enumerate}[(1)]
            \item % 2.1
                $
                    \left|\begin{array}{cccc}
                        2 & 1 \\
                        -4 & 4
                    \end{array}\right|
                    =
                    4\left|
                        \begin{array}{cccc}
                            2 & 1 \\
                            -1 & 1
                        \end{array}
                    \right|    
                $;
            \item % 2.2
                $
                    \left|
                        \begin{array}{cccc}
                            2 & 3 \\
                            -1 & 3
                        \end{array}
                    \right|
                    =
                    3\left|
                        \begin{array}{cccc}
                            2 & 1 \\
                            -1 & 1
                        \end{array}
                    \right|
                $.
        \end{enumerate}
    \item % 3
        $
            \left|
                \begin{array}{cccc}
                    3 & 2 \\
                    4 & 3
                \end{array}
            \right|
            =
            \left|
                \begin{array}{cccc}
                    3 & 2 \\
                    3 & 1
                \end{array}
            \right|
            +
            \left|
                \begin{array}{cccc}
                    3 & 2 \\
                    1 & 2
                \end{array}
            \right|
        $.
    \item % 4
        $
            \left|
                \begin{array}{cccc}
                    5 & -2 \\
                    2 & 1
                \end{array}
            \right|
            =
            -\left|
                \begin{array}{cccc}
                    -2 & 1 \\
                    5 & 2
                \end{array}
            \right|
        $.
    \item % 5
        $
            \left|
                \begin{array}{cccc}
                    2 & 3 \\
                    -3 & 1
                \end{array}
            \right|
            =
            \left|
                \begin{array}{cccc}
                    2 & -3 \\
                    3 & 1
                \end{array}
            \right|
        $.
    \item % 6
        $
            \left|
                \begin{array}{cccc}
                    a_{11} + b_{11} & a_{12} + b_{12} \\
                    a_{21} + b_{21} & a_{22} + b_{22}
                \end{array}
            \right|
            =
            \left|
                \begin{array}{cccc}
                    a_{11} & a_{12} \\
                    a_{21} & a_{22}
                \end{array}
            \right|
            +
            \left|
                \begin{array}{cccc}
                    a_{11} & b_{12} \\
                    a_{21} & b_{22}
                \end{array}
            \right|
            +
            \left|
                \begin{array}{cccc}
                    b_{11} & a_{12} \\
                    b_{21} & a_{22}
                \end{array}
            \right|
            +
            \left|
                \begin{array}{cccc}
                    b_{11} & b_{12} \\
                    b_{21} & b_{22}
                \end{array}
            \right|
        $.
\end{enumerate}
% \end{document}

        \section{三阶行列式}
            % % @author Shuning Zhang
% % @date 2020-03-16
% \documentclass[a4paper, 11pt]{ctexart}
% \usepackage{amsfonts, amsmath, amssymb, amsthm}
% \usepackage{color}
% \usepackage{enumerate}
% \usepackage[bottom=2cm, left=2.5cm, right=2.5cm, top=2cm]{geometry}
% \usepackage{multicol}
% \begin{document}
\begin{enumerate}
    \item % 1
        \begin{enumerate}[(1)]
            \item % 1.1
                $-8$;
            \item % 1.2
                $11$.
        \end{enumerate}
    \item % 2
        \begin{enumerate}[(1)]
            \item % 2.1
                $0$;
            \item % 2.2
                $22$.
        \end{enumerate}
    \item % 3
        \begin{enumerate}[(1)]
            \item % 3.1
                $0$;
            \item % 3.2
                $(x^2+1)e^x - x$.
        \end{enumerate}
    \item % 4
        \begin{enumerate}[(1)]
            \item % 4.1
                $2$;
            \item % 4.2
                $13$.
        \end{enumerate}
    \item % 5
        \begin{enumerate}[(1)]
            \item % 5.1
                $x_1 = -\dfrac{13}{5}$, $x_2 = -\dfrac{4}{5}$, $x_3 = -\dfrac{7}{5}$;
            \item % 5.2
                $x = 2$, $y = 0$, $z = -2$.
        \end{enumerate}
\end{enumerate}
% \end{document}

        \section{\texorpdfstring{$n$}{n} 阶行列式}
            % % @author Shuning Zhang
% % @date 2020-03-16
% \documentclass[a4paper, 11pt]{ctexart}
% \usepackage{amsfonts, amsmath, amssymb, amsthm}
% \usepackage{color}
% \usepackage{enumerate}
% \usepackage[bottom=2cm, left=2.5cm, right=2.5cm, top=2cm]{geometry}
% \usepackage{multicol}
% \begin{document}
\begin{enumerate}
    \item % 1
        略.
    \item % 2
        \begin{enumerate}[(1)]
            \item % 2.1
                $abcd$;
            \item % 2.2
                $-6$.
        \end{enumerate}
    \item % 3
        \begin{enumerate}[(1)]
            \item % 3.1
                $60$;
            \item % 3.2
                $2$.
        \end{enumerate}
    \item % 4
        \begin{enumerate}[(1)]
            \item % 4.1
                $0$;
            \item % 4.2
                $0$.
        \end{enumerate}
\end{enumerate}
% \end{document}

        \section{行列式的展开与转置}
            % % @author Shuning Zhang
% % @date 2020-03-17
% \documentclass[a4paper, 11pt]{ctexart}
% \usepackage{amsfonts, amsmath, amssymb, amsthm}
% \usepackage{color}
% \usepackage{enumerate}
% \usepackage[bottom=2cm, left=2.5cm, right=2.5cm, top=2cm]{geometry}
% \usepackage{multicol}
% \begin{document}
\begin{enumerate}
    \item % 1
        略.
    \item % 2
        略.
    \item % 3
        \begin{proof}
            \begin{align*}
                |A| &=
                \left|
                    \begin{array}{cccc}
                        0 & a_{12} & \cdots & a_{1n} \\
                        -a_{12} & 0 & \cdots & a_{2n} \\
                        \vdots & \vdots & \ddots & \vdots \\
                        -a_{1n} & -a_{2n} & \cdots & 0
                    \end{array}
                \right| \\
                &= (-1)
                \left|
                    \begin{array}{cccc}
                        0 & -a_{12} & \cdots & -a_{1n} \\
                        -a_{12} & 0 & \cdots & a_{2n} \\
                        \vdots & \vdots & \ddots & \vdots \\
                        -a_{1n} & -a_{2n} & \cdots & 0
                    \end{array}
                \right| \\
                &= (-1)^n
                \left|
                    \begin{array}{cccc}
                        0 & -a_{12} & \cdots & -a_{1n} \\
                        a_{12} & 0 & \cdots & -a_{2n} \\
                        \vdots & \vdots & \ddots & \vdots \\
                        a_{1n} & a_{2n} & \cdots & 0
                    \end{array}
                \right| \\
                &= (-1)^n|A'|.
            \end{align*}
            因为 $n$ 为奇数, 故 $|A| = -|A'|$, 因此 $|A| = 0$.
        \end{proof}
    \item % 4
        \begin{proof}
            \begin{align*}
                \left|
                    \begin{array}{ccccc}
                        0 & 0 & \cdots & 0 & b_1 \\
                        0 & 0 & \cdots & b_2 & 0 \\
                        \vdots & \vdots & \ddots & \vdots & \vdots \\
                        0 & b_{n-1} & \cdots & 0 & 0 \\
                        b_n & 0 & \cdots & 0 & 0
                    \end{array}
                \right|
                &= (-1)^{n-1}
                \left|
                    \begin{array}{ccccc}
                        b_1 & 0 & \cdots & 0 & 0 \\
                        0 & 0 & \cdots & 0 & b_2 \\
                        \vdots & \vdots & \ddots & \vdots & \vdots \\
                        0 & 0 & \cdots & 0 & 0 \\
                        0 & b_n & \cdots & 0 & 0
                    \end{array}
                \right| \\
                &= (-1)^{n-1}(-1)^{n-2} 
                \left|
                    \begin{array}{ccccc}
                        b_1 & 0 & \cdots & 0 & 0 \\
                        0 & b_2 & \cdots & 0 & 0 \\
                        \vdots & \vdots & \ddots & \vdots & \vdots \\
                        0 & 0 & \cdots & 0 & 0 \\
                        0 & 0 & \cdots & 0 & 0
                    \end{array}
                \right| \\
                &= (-1)^{\frac{(n-1)n}{2}}
                \left|
                    \begin{array}{ccccc}
                        b_1 & 0 & \cdots & 0 & 0 \\
                        0 & b_2 & \cdots & 0 & 0 \\
                        \vdots & \vdots & \ddots & \vdots & \vdots \\
                        0 & 0 & \cdots & b_{n-1} & 0 \\
                        0 & 0 & \cdots & 0 & b_n
                    \end{array}
                \right| \\
                &= (-1)^{\frac{(n-1)n}{2}}b_1b_2 \cdots b_n. \qedhere   
            \end{align*}
        \end{proof}
    \item % 5
        将行列式按第 $2$ 列进行展开, 则有
        \[
            f(x) = x
            \left|
                \begin{array}{cccc}
                    1 & 3 & 4 \\
                    -1 & -2 & -3 \\
                    -1 & -2 & -2
                \end{array}
            \right|
            + 2 \cdot A_{22} + 0 \cdot A_{32} + 7 \cdot A_{42}.    
        \]
        因此 $x$ 的系数为
        \[
            \left|
                \begin{array}{cccc}
                    1 & 3 & 4 \\
                    -1 & -2 & -3 \\
                    -1 & -2 & -2
                \end{array}
            \right|
            =
            \left|
                \begin{array}{cccc}
                    1 & 3 & 4 \\
                    0 & 1 & 1 \\
                    0 & 1 & 2
                \end{array}
            \right|
            =
            \left|
                \begin{array}{cccc}
                    1 & 3 & 4 \\
                    0 & 1 & 1 \\
                    0 & 0 & 1
                \end{array}
            \right|
            = 1. 
        \]
    \item % 6
        略.
\end{enumerate}
% \end{document}

        \section{行列式的计算}
        \section{行列式的等价定义}
        \section{Laplace 定理}
    \chapter{矩阵}
        \section{矩阵的概念}
        \section{矩阵的运算}
            % % @author Shuning Zhang
% % @date 2020-04-17
% \documentclass[a4paper, 11pt]{ctexart}
% \usepackage{amsfonts, amsmath, amssymb, amsthm}
% \usepackage{color}
% \usepackage{enumerate}
% \usepackage[bottom=2cm, left=2.5cm, right=2.5cm, top=2cm]{geometry}
% \usepackage{multicol}
% \begin{document}
\begin{enumerate}
    \item % 1
        略.
    \item % 2
        \begin{enumerate}[(1)]
            \item % 2.1
                $
                    \left(
                        \begin{array}{cccc}
                            1 & 5 \\
                            2 & 1
                        \end{array}
                    \right)
                $;
            \item % 2.2
                $
                    \left(
                        \begin{array}{cccc}
                            -2 & 2 \\
                            -2 & 0
                        \end{array}
                    \right)     
                $;
            \item % 2.3
                $
                    \left(
                        \begin{array}{cccc}
                            -1 & 10 \\
                            0 & 7
                        \end{array}
                    \right)
                $;
            \item % 2.4
                $
                    \left(
                        \begin{array}{cccc}
                            8 & -2 & 1 \\
                            -1 & 9 & 0 \\
                            -9 & -3 & 1 \\
                            -1 & 2 & 1
                        \end{array}
                    \right)
                $.
        \end{enumerate}
    \item % 3
        $
            AB =
            \left(
                \begin{array}{cccc}
                    13 & -1 \\
                    0 & -5
                \end{array}
            \right)
        $,
        $
            BA =
            \left(
                \begin{array}{cccc}
                    -1 & 1 & 3 \\
                    8 & -3 & 6 \\
                    4 & 0 & 12
                \end{array}
            \right)
        $.
    \item % 4
        \begin{enumerate}[(1)]
            \item % 4.1
                $
                    \left(
                        \begin{array}{cccc}
                            a^5 & 0 & 0 \\
                            0 & b^5 & 0 \\
                            0 & 0 & c^5
                        \end{array}
                    \right)
                $;
            \item % 4.2
                $
                    \left(
                        \begin{array}{cccc}
                            0 & 0 & 0 \\
                            0 & 0 & 0 \\
                            0 & 0 & 0
                        \end{array}
                    \right)
                $;
            \item % 4.3
                $
                    \left(
                        \begin{array}{cccc}
                            \cos{n\theta} & \sin{n\theta} \\
                            -\sin{n\theta} & \cos{n\theta}
                        \end{array}
                    \right)
                $.
        \end{enumerate}
    \item % 5
        $\displaystyle{
            xAx' = \sum_{i=1}^n a_{ii}x_i^2 + 2\sum_{\substack{i=1\\j>i}}^n a_{ij}x_ix_j
        }$.
    \item % 6
        略.
    \item % 7
        \begin{proof}
            \begin{align*}
                & (I_n - A)(I_n + A + A^2 + \cdots + A^{n-1}) \\
                ={} & (I_n - A)I_n + (I_n - A)A + (I_n - A)A^2 + \cdots + (I_n - A)A^{n-1} \\
                ={} & I_n \cdot I_n - A + A - A^2 + A^2 - A^3 + \cdots + A^{n-1} - A^n \\
                ={} & I_n - A^n \\
                ={} & I_n. \qedhere    
            \end{align*}
        \end{proof}
    \item % 8
        \begin{proof}
            用数学归纳法. 当阶数为 $2$ 时, 则有
            \[
                \left(
                    \begin{array}{cccc}
                        0 & 1 \\
                        0 & 0
                    \end{array}
                \right)
                \left(
                    \begin{array}{cccc}
                        0 & 1 \\
                        0 & 0
                    \end{array}
                \right)
                =
                \left(
                    \begin{array}{cccc}
                        0 & 0 \\
                        0 & 0
                    \end{array}
                \right).   
            \]
            假设阶数为 $n - 1$ 时, $A^{n-1} = 0$ (其中 $A$ 为 $n-1$ 阶方阵) 成立. 现考察 $A^n$ (其中 $A$ 为 $n$ 阶方阵). 由归纳假定, 可知
            \[
                A^{n-1} =
                \left(
                    \begin{array}{ccccc}
                        0 & 0 & 0 & \cdots & 0 \\
                        0 & 0 & 0 & \cdots & 0 \\
                        \vdots & \vdots & \vdots &  & \vdots \\
                        0 & 0 & 0 & \cdots & 1 \\
                        0 & 0 & 0 & \cdots & 0
                    \end{array}
                \right),
            \] 
            即 $A^{n-1}$ 除 $(n-1, n)$ 元素为 $1$ 之外, 其余元素均为 $0$. 因此 $A^{n-1}A = A^n = 0$.
        \end{proof}
    \item % 9
        \begin{proof}
            设
            \[
                A =
                \left(
                    \begin{array}{cccc}
                        a_{11} & a_{12} & \cdots & a_{1n} \\
                        a_{12} & a_{22} & \cdots & a_{2n} \\
                        \vdots & \vdots &  & \vdots \\
                        a_{1n} & a_{2n} & \cdots & a_{nn}
                    \end{array}
                \right).    
            \]
            那么
            \[
                A^2 =
                \left(
                    \begin{array}{cccc}
                        a_{11}^2 + \cdots + a_{1n}^2 & * & \cdots & * \\
                        * & a_{12}^2 + \cdots + a_{2n}^2 & \cdots & * \\
                        \vdots & \vdots &  & \vdots \\
                        * & * & \cdots & a_{1n}^2 + \cdots + a_{nn}^2
                    \end{array}
                \right),
            \]
            即 $a_{11}^2 + \cdots + a_{1n}^2 = a_{12}^2 + \cdots + a_{2n}^2 = \cdots = a_{1n}^2 + \cdots + a_{nn}^2 = 0$. 因此 $A = 0$.
        \end{proof}
    \item % 10
        \begin{proof}
            \begin{align*}
                & \left(
                    \begin{array}{cccc}
                        a_{11} & \cdots & a_{1n} \\
                        \vdots &  & \vdots \\
                        a_{n1} & \cdots & a_{nn}
                    \end{array}
                \right) \\
                ={} &
                \left(
                    \begin{array}{cccc}
                        a_{11} & \cdots & \dfrac{a_{1n}+a_{n1}}{2} \\
                        \vdots &  & \vdots \\
                        \dfrac{a_{1n}+a_{n1}}{2} & \cdots & a_{nn}
                    \end{array}
                \right)
                +
                \left(
                    \begin{array}{cccc}
                        0 & \cdots & \dfrac{a_{1n}-a_{n1}}{2} \\
                        \vdots &  & \vdots \\
                        -\dfrac{a_{1n}-a_{n1}}{2} & \cdots & 0
                    \end{array}
                \right). \qedhere 
            \end{align*}
        \end{proof}
    \item % 11
        结论是显然的, 运用矩阵乘法的规则即可证明.
    \item % 12
        \begin{enumerate}[(1)]
            \item % 12.1
                设
                \[
                    B =
                    \left(
                        \begin{array}{cccc}
                            a & b \\
                            c & d
                        \end{array}
                    \right).    
                \]
                那么
                \[
                    BA =
                    \left(
                        \begin{array}{cccc}
                            a & a+b \\
                            c & c+d
                        \end{array}
                    \right),\ 
                    AB =
                    \left(
                        \begin{array}{cccc}
                            a+c & b+d \\
                            c & d
                        \end{array}
                    \right).
                \]
                由 $BA = AB$, 解得
                \[
                    B =
                    \left(
                        \begin{array}{cccc}
                            x & y \\
                            0 & x
                        \end{array}
                    \right).
                \]
                其中 $x, y \in \mathrm{R}$.
            \item % 12.2
                同 (1).
        \end{enumerate}
    \item % 13
        \begin{enumerate}[(1)]
            \item % 13.1
                \begin{proof}
                    必要性. $AB$ 的第 $(i, j)$ 元为
                    \[
                        \sum_{k=1}^n a_{ik}b_{kj}
                    \]
                    因为 $AB$ 是对称阵, 故 $(i, j)$ 元就是 $(j, i)$ 元, 即
                    \[
                        \sum_{k=1}^n a_{ik}b_{kj} = \sum_{k=1}^n a_{jk}b_{ki}.    
                    \]
                    $BA$ 的 $(i, j)$ 元为
                    \[
                        \sum_{k=1}^n b_{ik}a_{kj},    
                    \]
                    因为 $A$, $B$ 都是对称阵, 故 $b_{ik} = b_{ki}$, $a_{kj} = a_{jk}$, 故 $BA$ 的 $(i, j)$ 元也可写为
                    \[
                        \sum_{k=1}^n a_{jk}b_{ki}.   
                    \]
                    这正是 $AB$ 的 $(j, i)$ 元也即是 $AB$ 的 $(i, j)$ 元. 因此 $AB = BA$.
                    
                    充分性反推回去即可.
                \end{proof}
            \item % 13.2
                \begin{proof}
                    记 $AB(i, j)$ 为 $AB$ 的第 $(i, j)$ 元素, 那么
                    \[
                        AB(i, j) = \sum_{k=1}^n a_{ik}b_{kj},   
                    \]
                    又 $AB$ 是反对称阵, 故
                    \[
                        AB(i, j) = -AB(j, i) = -\sum_{k=1}^n a_{jk}b_{ki} = \sum_{k=1}^n a_{jk}(-b_{ki}),  
                    \]
                    又 $A$ 是对称阵, $B$ 是同阶的反对称阵, 故 $a_{jk} = a_{kj}$, $-b_{ki} = b_{ik}$, 也就是
                    \[
                        \sum_{k=1}^n a_{jk}(-b_{ki}) = \sum_{k=1}^n b_{ik}a_{kj} = BA(i, j).    
                    \]
                    因此 $AB(i, j) = BA(j, i)$, 即 $AB = BA$.
                \end{proof}
        \end{enumerate}
    \item % 14
    \item % 15
\end{enumerate}
% \end{document}

        \section{方阵的逆阵}
            % @author Shuning Zhang
% @date 2020-04-17
\documentclass[a4paper, 11pt]{ctexart}
\usepackage{amsfonts, amsmath, amssymb, amsthm}
\usepackage{color}
\usepackage{enumerate}
\usepackage[bottom=2cm, left=2.5cm, right=2.5cm, top=2cm]{geometry}
\usepackage{multicol}
\begin{document}
\begin{enumerate}
    \item % 1
        \begin{enumerate}[(1)]
            \item % 1.1
                $
                    \dfrac13
                    \left(
                        \begin{array}{ccc}
                            5 & -2 & -1 \\
                            -1 & 1 & 2 \\
                            1 & -1 & 1
                        \end{array}
                    \right)
                $;
            \item % 1.2
                $
                    \left(
                        \begin{array}{cccc}
                            a_1^{-1} & 0 & \cdots & 0 \\
                            0 & a_2^{-1} & \cdots & 0 \\
                            \vdots & \vdots &  & \vdots \\
                            0 & 0 & \cdots & a_n^{-1}
                        \end{array}
                    \right)
                $.
        \end{enumerate}
    \item % 2
        $
            X =
            \left(
                \begin{array}{ccc}
                    -4 & 2 & 1 \\
                    4 & -1 & 2 \\
                    3 & -1 & 1
                \end{array}
            \right)
            \left(
                \begin{array}{c}
                    8 \\
                    11 \\
                    -11
                \end{array}
            \right)
            =
            \left(
                \begin{array}{c}
                    1 \\
                    -1 \\
                    2
                \end{array}
            \right)
        $.
    \item % 3
        \begin{proof}
            \[
                (A^k)^{-1} = \underbrace{AA \cdots A}_{k\ \text{项}} = \underbrace{A^{-1}A^{-1}\cdots A^{-1}}_{k\ \text{项}} = (A^{-1})^k. \qedhere   
            \]
        \end{proof}
    \item % 4
        \begin{proof}
            设 $A$ 的逆矩阵为 $A^{-1}$, 那么
            \begin{align*}
                A^{-1}AB = A^{-1}AC &\Leftrightarrow B = C, \\
                BAA^{-1} = CAA^{-1} &\Leftrightarrow B = C. \qedhere
            \end{align*}
        \end{proof}
    \item % 5
        提示: 定理 2.3.1 的逆否命题.
    \item % 6
    \item % 7
    \item % 8
    \item % 9
    \item % 10
\end{enumerate}
\end{document}

        \section{矩阵的初等变换与初等矩阵}
        \section{矩阵乘积的行列式与初等变换法求逆阵}
        \section{分块矩阵}
        \section{Cauchy-Binet 公式}
    \chapter{线性空间}
        \section{数域}
        \section{行向量和列向量}
        \section{线性空间}
        \section{向量的线性关系}
        \section{向量组的秩}
        \section{矩阵的秩}
        \section{坐标向量}
        \section{基变换与过渡矩阵}
        \section{子空间}
        \section{线性方程组的解}
    \chapter{线性映射}
        \section{线性映射的概念}
        \section{线性映射的运算}
        \section{线性映射与矩阵}
        \section{线性映射的像与核}
        \section{不变子空间}
    \chapter{多项式}
        \section{一元多项式代数}
        \section{整除}
        \section{最大公因式}
        \section{因式分解}
        \section{多项式函数}
        \section{复系数多项式}
        \section{实系数多项式和有理系数多项式}
        \section{多元多项式}
        \section{对称多项式}
        \section{结式和判别式}
    \chapter{特征值}
        \section{特征值和特征向量}
        \section{对角化}
        \section{极小多项式与 Cayley-Hamilton 定理}
        \section{特征值的估计}
    \chapter{相似标准型}
        \section{多项式矩阵}
        \section{矩阵的法式}
        \section{不变因子}
        \section{有理标准型}
        \section{初等因子}
        \section{Jordan 标准型}
        \section{Jordan 标准型的进一步讨论和应用}
        \section{矩阵函数}
    \chapter{二次型}
        \section{二次型的化简与矩阵的合同}
        \section{二次型的化简}
        \section{惯性定理}
        \section{正定型与正定矩阵}
        \section{Hermite 型}
    \chapter{内积空间}
        \section{内积空间的概念}
        \section{内积的表示和正交基}
        \section{伴随}
        \section{内积空间的同构, 正交变换和酉变换}
        \section{自伴随算子}
        \section{复正规算子}
        \section{实正规矩阵}
        \section{谱}
        \section{奇异值分解}
        \section{最小二乘解}
    \chapter{双线性型}
        \section{对偶空间}
        \section{双线性型}
        \section{纯量积}
        \section{交错型与辛空间}
        \section{对称型与正交几何}
\end{document}
