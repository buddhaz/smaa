% % @author Shuning Zhang
% % @date 2020-04-17
% \documentclass[a4paper, 11pt]{ctexart}
% \usepackage{amsfonts, amsmath, amssymb, amsthm}
% \usepackage{color}
% \usepackage{enumerate}
% \usepackage[bottom=2cm, left=2.5cm, right=2.5cm, top=2cm]{geometry}
% \usepackage{multicol}
% \begin{document}
\begin{enumerate}
    \item % 1
        略.
    \item % 2
        \begin{enumerate}[(1)]
            \item % 2.1
                $
                    \left(
                        \begin{array}{cccc}
                            1 & 5 \\
                            2 & 1
                        \end{array}
                    \right)
                $;
            \item % 2.2
                $
                    \left(
                        \begin{array}{cccc}
                            -2 & 2 \\
                            -2 & 0
                        \end{array}
                    \right)     
                $;
            \item % 2.3
                $
                    \left(
                        \begin{array}{cccc}
                            -1 & 10 \\
                            0 & 7
                        \end{array}
                    \right)
                $;
            \item % 2.4
                $
                    \left(
                        \begin{array}{cccc}
                            8 & -2 & 1 \\
                            -1 & 9 & 0 \\
                            -9 & -3 & 1 \\
                            -1 & 2 & 1
                        \end{array}
                    \right)
                $.
        \end{enumerate}
    \item % 3
        $
            AB =
            \left(
                \begin{array}{cccc}
                    13 & -1 \\
                    0 & -5
                \end{array}
            \right)
        $,
        $
            BA =
            \left(
                \begin{array}{cccc}
                    -1 & 1 & 3 \\
                    8 & -3 & 6 \\
                    4 & 0 & 12
                \end{array}
            \right)
        $.
    \item % 4
        \begin{enumerate}[(1)]
            \item % 4.1
                $
                    \left(
                        \begin{array}{cccc}
                            a^5 & 0 & 0 \\
                            0 & b^5 & 0 \\
                            0 & 0 & c^5
                        \end{array}
                    \right)
                $;
            \item % 4.2
                $
                    \left(
                        \begin{array}{cccc}
                            0 & 0 & 0 \\
                            0 & 0 & 0 \\
                            0 & 0 & 0
                        \end{array}
                    \right)
                $;
            \item % 4.3
                $
                    \left(
                        \begin{array}{cccc}
                            \cos{n\theta} & \sin{n\theta} \\
                            -\sin{n\theta} & \cos{n\theta}
                        \end{array}
                    \right)
                $.
        \end{enumerate}
    \item % 5
        $\displaystyle{
            xAx' = \sum_{i=1}^n a_{ii}x_i^2 + 2\sum_{\substack{i=1\\j>i}}^n a_{ij}x_ix_j
        }$.
    \item % 6
        略.
    \item % 7
        \begin{proof}
            \begin{align*}
                & (I_n - A)(I_n + A + A^2 + \cdots + A^{n-1}) \\
                ={} & (I_n - A)I_n + (I_n - A)A + (I_n - A)A^2 + \cdots + (I_n - A)A^{n-1} \\
                ={} & I_n \cdot I_n - A + A - A^2 + A^2 - A^3 + \cdots + A^{n-1} - A^n \\
                ={} & I_n - A^n \\
                ={} & I_n. \qedhere    
            \end{align*}
        \end{proof}
    \item % 8
        \begin{proof}
            用数学归纳法. 当阶数为 $2$ 时, 则有
            \[
                \left(
                    \begin{array}{cccc}
                        0 & 1 \\
                        0 & 0
                    \end{array}
                \right)
                \left(
                    \begin{array}{cccc}
                        0 & 1 \\
                        0 & 0
                    \end{array}
                \right)
                =
                \left(
                    \begin{array}{cccc}
                        0 & 0 \\
                        0 & 0
                    \end{array}
                \right).   
            \]
            假设阶数为 $n - 1$ 时, $A^{n-1} = 0$ (其中 $A$ 为 $n-1$ 阶方阵) 成立. 现考察 $A^n$ (其中 $A$ 为 $n$ 阶方阵). 由归纳假定, 可知
            \[
                A^{n-1} =
                \left(
                    \begin{array}{ccccc}
                        0 & 0 & 0 & \cdots & 0 \\
                        0 & 0 & 0 & \cdots & 0 \\
                        \vdots & \vdots & \vdots &  & \vdots \\
                        0 & 0 & 0 & \cdots & 1 \\
                        0 & 0 & 0 & \cdots & 0
                    \end{array}
                \right),
            \] 
            即 $A^{n-1}$ 除 $(n-1, n)$ 元素为 $1$ 之外, 其余元素均为 $0$. 因此 $A^{n-1}A = A^n = 0$.
        \end{proof}
    \item % 9
        \begin{proof}
            设
            \[
                A =
                \left(
                    \begin{array}{cccc}
                        a_{11} & a_{12} & \cdots & a_{1n} \\
                        a_{12} & a_{22} & \cdots & a_{2n} \\
                        \vdots & \vdots &  & \vdots \\
                        a_{1n} & a_{2n} & \cdots & a_{nn}
                    \end{array}
                \right).    
            \]
            那么
            \[
                A^2 =
                \left(
                    \begin{array}{cccc}
                        a_{11}^2 + \cdots + a_{1n}^2 & * & \cdots & * \\
                        * & a_{12}^2 + \cdots + a_{2n}^2 & \cdots & * \\
                        \vdots & \vdots &  & \vdots \\
                        * & * & \cdots & a_{1n}^2 + \cdots + a_{nn}^2
                    \end{array}
                \right),
            \]
            即 $a_{11}^2 + \cdots + a_{1n}^2 = a_{12}^2 + \cdots + a_{2n}^2 = \cdots = a_{1n}^2 + \cdots + a_{nn}^2 = 0$. 因此 $A = 0$.
        \end{proof}
    \item % 10
        \begin{proof}
            \begin{align*}
                & \left(
                    \begin{array}{cccc}
                        a_{11} & \cdots & a_{1n} \\
                        \vdots &  & \vdots \\
                        a_{n1} & \cdots & a_{nn}
                    \end{array}
                \right) \\
                ={} &
                \left(
                    \begin{array}{cccc}
                        a_{11} & \cdots & \dfrac{a_{1n}+a_{n1}}{2} \\
                        \vdots &  & \vdots \\
                        \dfrac{a_{1n}+a_{n1}}{2} & \cdots & a_{nn}
                    \end{array}
                \right)
                +
                \left(
                    \begin{array}{cccc}
                        0 & \cdots & \dfrac{a_{1n}-a_{n1}}{2} \\
                        \vdots &  & \vdots \\
                        -\dfrac{a_{1n}-a_{n1}}{2} & \cdots & 0
                    \end{array}
                \right). \qedhere 
            \end{align*}
        \end{proof}
    \item % 11
        结论是显然的, 运用矩阵乘法的规则即可证明.
    \item % 12
        \begin{enumerate}[(1)]
            \item % 12.1
                设
                \[
                    B =
                    \left(
                        \begin{array}{cccc}
                            a & b \\
                            c & d
                        \end{array}
                    \right).    
                \]
                那么
                \[
                    BA =
                    \left(
                        \begin{array}{cccc}
                            a & a+b \\
                            c & c+d
                        \end{array}
                    \right),\ 
                    AB =
                    \left(
                        \begin{array}{cccc}
                            a+c & b+d \\
                            c & d
                        \end{array}
                    \right).
                \]
                由 $BA = AB$, 解得
                \[
                    B =
                    \left(
                        \begin{array}{cccc}
                            x & y \\
                            0 & x
                        \end{array}
                    \right).
                \]
                其中 $x, y \in \mathrm{R}$.
            \item % 12.2
                同 (1).
        \end{enumerate}
    \item % 13
        \begin{enumerate}[(1)]
            \item % 13.1
                \begin{proof}
                    必要性. $AB$ 的第 $(i, j)$ 元为
                    \[
                        \sum_{k=1}^n a_{ik}b_{kj}
                    \]
                    因为 $AB$ 是对称阵, 故 $(i, j)$ 元就是 $(j, i)$ 元, 即
                    \[
                        \sum_{k=1}^n a_{ik}b_{kj} = \sum_{k=1}^n a_{jk}b_{ki}.    
                    \]
                    $BA$ 的 $(i, j)$ 元为
                    \[
                        \sum_{k=1}^n b_{ik}a_{kj},    
                    \]
                    因为 $A$, $B$ 都是对称阵, 故 $b_{ik} = b_{ki}$, $a_{kj} = a_{jk}$, 故 $BA$ 的 $(i, j)$ 元也可写为
                    \[
                        \sum_{k=1}^n a_{jk}b_{ki}.   
                    \]
                    这正是 $AB$ 的 $(j, i)$ 元也即是 $AB$ 的 $(i, j)$ 元. 因此 $AB = BA$.
                    
                    充分性反推回去即可.
                \end{proof}
            \item % 13.2
                \begin{proof}
                    记 $AB(i, j)$ 为 $AB$ 的第 $(i, j)$ 元素, 那么
                    \[
                        AB(i, j) = \sum_{k=1}^n a_{ik}b_{kj},   
                    \]
                    又 $AB$ 是反对称阵, 故
                    \[
                        AB(i, j) = -AB(j, i) = -\sum_{k=1}^n a_{jk}b_{ki} = \sum_{k=1}^n a_{jk}(-b_{ki}),  
                    \]
                    又 $A$ 是对称阵, $B$ 是同阶的反对称阵, 故 $a_{jk} = a_{kj}$, $-b_{ki} = b_{ik}$, 也就是
                    \[
                        \sum_{k=1}^n a_{jk}(-b_{ki}) = \sum_{k=1}^n b_{ik}a_{kj} = BA(i, j).    
                    \]
                    因此 $AB(i, j) = BA(j, i)$, 即 $AB = BA$.
                \end{proof}
        \end{enumerate}
    \item % 14
    \item % 15
\end{enumerate}
% \end{document}
