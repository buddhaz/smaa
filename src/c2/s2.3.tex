% % @author Shuning Zhang
% % @date 2020-04-19
% \documentclass[a4paper, 11pt]{ctexart}
% \usepackage{amsfonts, amsmath, amssymb, amsthm}
% \usepackage{color}
% \usepackage{enumerate}
% \usepackage[bottom=2cm, left=2.5cm, right=2.5cm, top=2cm]{geometry}
% \usepackage{multicol}
% \begin{document}
\begin{enumerate}
    \item % 1
        \begin{enumerate}[(1)]
            \item % 1.1
                $
                    \dfrac13
                    \left(
                        \begin{array}{ccc}
                            5 & -2 & -1 \\
                            -1 & 1 & 2 \\
                            1 & -1 & 1
                        \end{array}
                    \right)
                $;
            \item % 1.2
                $
                    \left(
                        \begin{array}{cccc}
                            a_1^{-1} & 0 & \cdots & 0 \\
                            0 & a_2^{-1} & \cdots & 0 \\
                            \vdots & \vdots &  & \vdots \\
                            0 & 0 & \cdots & a_n^{-1}
                        \end{array}
                    \right)
                $.
        \end{enumerate}
    \item % 2
        $
            X =
            \left(
                \begin{array}{ccc}
                    -4 & 2 & 1 \\
                    4 & -1 & 2 \\
                    3 & -1 & 1
                \end{array}
            \right)
            \left(
                \begin{array}{c}
                    8 \\
                    11 \\
                    -11
                \end{array}
            \right)
            =
            \left(
                \begin{array}{c}
                    1 \\
                    -1 \\
                    2
                \end{array}
            \right)
        $.
    \item % 3
        \begin{proof}
            \[
                (A^k)^{-1} = \underbrace{AA \cdots A}_{k\ \text{项}} = \underbrace{A^{-1}A^{-1}\cdots A^{-1}}_{k\ \text{项}} = (A^{-1})^k. \qedhere   
            \]
        \end{proof}
    \item % 4
        \begin{proof}
            设 $A$ 的逆矩阵为 $A^{-1}$, 那么
            \begin{align*}
                A^{-1}AB = A^{-1}AC &\Leftrightarrow B = C, \\
                BAA^{-1} = CAA^{-1} &\Leftrightarrow B = C. \qedhere
            \end{align*}
        \end{proof}
    \item % 5
        提示: 定理 2.3.1 的逆否命题.
    \item % 6
        \begin{proof}
            因为
            \[
                (I_n-A)(I_n^{m-1}+I_n^{m-2}A + \cdots + I_nA^{m-2} + A^{m-1}) = I_n^m - A^m = I_n,    
            \]
            同理有
            \[
                (I_n^{m-1}+I_n^{m-2}A + \cdots + I_nA^{m-2} + A^{m-1})(I_n-A) = I_n.   
            \]
            因此 $I_n-A$ 是非奇异阵.
        \end{proof}
    \item % 7
        \begin{proof}
            \begin{align*}
                B(A+B)^{-1}A(A^{-1}+B^{-1}) &= B(A+B)^{-1}(I_n+AB^{-1}) \\
                &= B(A+B)^{-1}(BB^{-1}+AB^{-1}) \\
                &= B(A+B)^{-1}(A+B)B^{-1} \\
                &= BB^{-1} \\
                &= I_n. \qedhere
            \end{align*}
        \end{proof}
    \item % 8
        \begin{proof}
            \begin{align*}
                (A + I_n)^2 &= (A + I_n)(A + I_n) \\
                &= A^2 + A + A + I_n^2 \\
                &= I_n + A + A + I_n \\
                &= 2(A + I_n),   
            \end{align*}
            故
            \begin{align*}
                (A + I_n)^2 = 2(A + I_n) &\Leftrightarrow (A+I_n)(A+I_n) = 2I_n(A+I_n),
            \end{align*}
            由第 4 题可得
            \[
                A+I_n = 2I_n = I_n + I_n,   
            \]
            因此 $A = I_n$.
        \end{proof}
    \item % 9
        \begin{proof}
            \begin{align*}
                A^2=A &\Leftrightarrow A^2 - A - 2I_n = -2I_n \\
                &\Leftrightarrow (A+I_n)(A-2I_n) = -2I_n. \qedhere
            \end{align*}
        \end{proof}
    \item % 10
        \begin{proof}
            \begin{align*}
                A^2 - A - 3I_n = 0 &\Leftrightarrow A^2 - A - 2I_n = I_n \\
                &\Leftrightarrow (A+I_n)(A-2I_n) = I_n,
            \end{align*}
            同理有 $(A-2I_n)(A+I_n) = I_n$.
            因此 $A-2I_n$ 是非奇异阵.
        \end{proof}
\end{enumerate}
% \end{document}
