% @author Shuning Zhang
% @date 2020-04-17
\documentclass[a4paper, 11pt]{ctexart}
\usepackage{amsfonts, amsmath, amssymb, amsthm}
\usepackage{color}
\usepackage{enumerate}
\usepackage[bottom=2cm, left=2.5cm, right=2.5cm, top=2cm]{geometry}
\usepackage{multicol}
\begin{document}
\begin{enumerate}
    \item % 1
        \begin{enumerate}[(1)]
            \item % 1.1
                $
                    \dfrac13
                    \left(
                        \begin{array}{ccc}
                            5 & -2 & -1 \\
                            -1 & 1 & 2 \\
                            1 & -1 & 1
                        \end{array}
                    \right)
                $;
            \item % 1.2
                $
                    \left(
                        \begin{array}{cccc}
                            a_1^{-1} & 0 & \cdots & 0 \\
                            0 & a_2^{-1} & \cdots & 0 \\
                            \vdots & \vdots &  & \vdots \\
                            0 & 0 & \cdots & a_n^{-1}
                        \end{array}
                    \right)
                $.
        \end{enumerate}
    \item % 2
        $
            X =
            \left(
                \begin{array}{ccc}
                    -4 & 2 & 1 \\
                    4 & -1 & 2 \\
                    3 & -1 & 1
                \end{array}
            \right)
            \left(
                \begin{array}{c}
                    8 \\
                    11 \\
                    -11
                \end{array}
            \right)
            =
            \left(
                \begin{array}{c}
                    1 \\
                    -1 \\
                    2
                \end{array}
            \right)
        $.
    \item % 3
        \begin{proof}
            \[
                (A^k)^{-1} = \underbrace{AA \cdots A}_{k\ \text{项}} = \underbrace{A^{-1}A^{-1}\cdots A^{-1}}_{k\ \text{项}} = (A^{-1})^k. \qedhere   
            \]
        \end{proof}
    \item % 4
        \begin{proof}
            设 $A$ 的逆矩阵为 $A^{-1}$, 那么
            \begin{align*}
                A^{-1}AB = A^{-1}AC &\Leftrightarrow B = C, \\
                BAA^{-1} = CAA^{-1} &\Leftrightarrow B = C. \qedhere
            \end{align*}
        \end{proof}
    \item % 5
        提示: 定理 2.3.1 的逆否命题.
    \item % 6
    \item % 7
    \item % 8
    \item % 9
    \item % 10
\end{enumerate}
\end{document}
